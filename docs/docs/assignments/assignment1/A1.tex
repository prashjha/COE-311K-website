\documentclass[12pt,a4paper]{report}
\usepackage[utf8]{inputenc}
\usepackage[T1]{fontenc}
\usepackage{amsmath}
\usepackage{amsfonts}
\usepackage{amssymb}
\usepackage{graphicx}

\usepackage[a4paper, total={6.5in, 8.5in}]{geometry}

\begin{document}

\begin{center} 
\textbf{Assignment 1} 
\end{center}

Consider a problem of computing the velocity of a rocket. Let $m = m(t)$ kg is the mass of rocket at time $t$ s (seconds), $0\leq t \leq t_F$. Let rocket is moving vertically upwards and $v = v(t)$ m/s is it's velocity (positive means upward velocity and negative means downward). 

To increase the velocity of a rocket and overcome the forces due to earth's gravity and drag, mass is ejected from a rocket at rapid speed. Let $\dot{m}_e = \dot{m}_e(t)$ is the rate of the mass leaving the rocket at time $t$ (in units of $kg/s$). Also, suppose that the mass leaves rocket at a relative speed $v_e$ m/s. $v_e$ is negative, i.e., mass is ejected downwards (towards surface of earth) relative to the rocket, and, therefore, due to the Newton's third law, rocket moves upward (so the rocket feels the thrust).

The following two equations, based on the \emph{conservation of mass} and \emph{conservation of linear momentum}, define the mass and velocity of the rocket at any time $t$, $0< t \leq t_F$,
\begin{align}
\frac{d m(t)}{d t} &= - \dot{m}_e(t), \label{eq:ode1} \\
\frac{d v(t)}{d t} &= \frac{\dot{m}_e}{m(t)} (v(t) - v_e) - g - \frac{c_d}{m(t)} |v(t)| \, v(t) \label{eq:ode2}.
\end{align}
Initial conditions are
\begin{equation}
m(0) = m_0, \qquad v(0) = v_0,
\end{equation}
where $m_0, v_0$ are the known numbers. 

While in reality the mass ejection is done in different stages, here we will consider only stage 1, and define the mass ejection rate function $\dot{m}_e$ as follows:
\begin{equation}\label{eq:me}
\dot{m}_e(t) = \begin{cases}
\bar{m}, \qquad \text{if } 0 \leq t \leq \bar{t}, \\
0, \qquad \text{otherwise} .
\end{cases}
\end{equation}
Thus, we have mass ejection at a constant rate, $\bar{m}$, from $t=0$ to $t=\bar{t}$, and then no ejection.

\textit{Remark 1.} The derivation of above equations is in a file supplement to this assignment file. 

\textit{Parameter values.} Take $t_F = 10$ s, $m_0 = 1000$ kg, $v_0 = 0$ m/s, $\bar{m} = 100$ kg/s, $v_e = -2000$ m/s, $c_d = 0.1$ m/kg, $g = 9.81$ m/s$^2$, and $\bar{t} = 0.4 t_F$ s. Further, consider discrete times between $0$ and $t_F$ with spacing $\Delta t = 0.125$ s. 

\vspace{10pt}

\textbf{Problem 1 (50 marks).} Write down the numerical approximation of \eqref{eq:ode1} and \eqref{eq:ode2} (similar to the gravity problem worked out in the class but now you also have to solve for the mass), and 
compute $m(t_F)$ and $v(t_F)$ using the parameters specified above. Also, plot the values of $m(t_i)$ and separately plot $v(t_i)$ at discrete times $t_i$ using MATLAB plot function. 

\vspace{10pt}
\textit{Remark 2.} If you take $\Delta t = 0.5$ s and all other parameters as described above, $m(t_F)$ and $v(t_F)$ will be about 600 kg and 456 m/s, respectively. This should help you in checking your code.

\vspace{10pt}
\textbf{Problem 2 (20 marks).} Run your code with four different $\Delta t = 0.5, 0.25, 0.125, 0.0625$ and list the values of $m(t_F)$ and $v(t_F)$ for each case. 

\vspace{10pt}
\textbf{Problem 3 (25 marks).} Instead of a `step' function for $\dot{m}_e$ in \eqref{eq:me}, try another function, say a function that linearly increases from $0$ to $\bar{m}$ from time $0$ to $\bar{t}$, and, for time above $\bar{t}$, $\dot{m}_e(t) = 0$. Using either the new function I just described or your own new function for $\dot{m}_e$, compute $m(t_F)$ and $v(t_F)$ with parameters listed above and with $\Delta t = 0.125$. Compare with the result for the `step' function.

\vspace{10pt}
\textbf{Problem 4 (5 marks).} In our derivation of \eqref{eq:ode2}, we assumed that the rocket is always vertical. Comment on what may change if suppose the rocket is at a fixed angle $\theta$ from the vertical direction (so if $\theta = 0$ degrees, the rocket will be vertical). (Hint: something to do with the gravity force.)
 
\end{document}